%%%%%%%%%%%%%%%%%%%%%%%%%%%%%%%%%%%%%%%%%%%%%%%%%%%%%%
% \documentclass[11pt,oneside,a4paper, draft]{article}
\documentclass[12pt,oneside,a4paper]{article}
% \documentclass[11pt,oneside,a4paper]{letter}
\usepackage[a4paper, margin=1in]{geometry}   % MARGENS

%%%%%%%%%%%%%%%%%%%%%%%%%%%%%%%%%%%%%%%%%%%%%%%%%%%%%%
% Pacotes básicos 
% \usepackage[T1]{fontenc}		% Selecao de codigos de fonte.
\usepackage[0T1]{fontenc}		% Selecao de codigos de fonte.
\usepackage[utf8]{inputenc}		% Codificacao do documento (conversão automática dos acentos)
\usepackage[english]{babel}
% \selectlanguage{english}

%%%%%%%%%%%%%%%%%%%%%%%%%%%%%%%%%%%%%%%%%%%%%%%%%%%%%%
% Pacotes de citações
% BIBLIOGRAPHYSTYLE (BST) DIFERENTE LA EM BAIXO
% \usepackage[alf]{abntex2cite}	% Citações padrão ABNT 
\usepackage{natbib}	% Citações}

%%%%%%%%%%%%%%%%%%%%%%%%%%%%%%%%%%%%%%%%%%%%%%%%%%%%%%
% \usepackage{lastpage}			% Usado pela Ficha catalográfica
\usepackage{indentfirst}		% Indenta o primeiro parágrafo de cada seção.
\usepackage{color}				% Controle das cores
\usepackage{graphicx}			% Inclusão de gráficos
\usepackage{microtype} 			% para melhorias de justificação
\usepackage{mathtools, amsmath, amssymb, amsthm, latexsym}
\usepackage{lscape}				% Gira a página em 90 graus
%\usepackage{listings}			% Formatação para inserir códigos
\usepackage[normalem]{ulem}
\usepackage[all]{xy}
\usepackage{xcolor}
\usepackage{ragged2e}           % formatação texto
\usepackage{bm}                 % bold symbols 
\usepackage[colorlinks, citecolor=blue,urlcolor=blue]{hyperref} % referencias dentro do texo (*QUEBRA MEMOIR*)
\usepackage{url}                % URL
\usepackage{graphicx}			% Inclusão de gráficos
\usepackage{subcaption}			% Faz subfiguras
\usepackage{textcase}			% MakeTextUppercase
%\usepackage{subfigure}         % subfigures
% \usepackage{setspace}         % Espaçamento
\usepackage{pdfpages}           % inclui páginas de pdfs (*FICHA CATALOGRAFICA*)
\usepackage[flushleft]{threeparttable} % notas nas tabelas
\usepackage{enumitem}
% \usepackage{titlesec}
\usepackage{titling}          % personalized other things
\usepackage{fancyhdr}         % personalized page style
\usepackage{lipsum}           % dummy text
% \usepackage[displaymath, pagewise]{lineno}           % show line numbers
\usepackage[]{lineno}           % show line numbers
% \linenumbers

%%%%%%%%%%%%%%%%%%%%%%%%%%%%%%%%%%%%%%%%%%%%%%%%%%%%%%
\input{macros.tex}
% \pagestyle{headings}
\pagestyle{fancy}

%%%%%%%%%%%%%%%%%%%%%%%%%%%%%%%%%%%%%%%%%%%%%%%%%%%%%%
% Begin
\begin{document}

%%%%%%%%%%%%%%%%%%%%%%%%%%%%%%%%%%%%%%%%%%%%%%%%%%%%%%
% METADATA
\title{Notes on Mean Testing}
\author{Paulo F. Naibert}
\date{\today}
% \thanks{} \\ 
%email\mailto{sth sthe}

%%%%%%%%%%%%%%%%%%%%%%%%%%%%%%%%%%%%%%%%%%%%%%%%%%%%%%
% TITLE PAGE
\maketitle
%%%%%%%%%%%%%%%%%%%%%%%%%%%%%%%%%%%%%%%%%%%%%%%%%%%%%%

\thispagestyle{headings}

%%%%%%%%%%%%%%%%%%%%%%%%%%%%%%%%%%%%%%%%%%%%%%%%%%%%%%
% MEAN TESTING
\section{Mean Testing}
Based on \cite{lw2008-sr, lw2011-var}

$T$ pairs: $(r_{1i}, r_{1n})', \dots, (r_{Ti}, r_{Tn})'$.

% pairs as matrix
% $T \times 2$ matrix:
% \begin{align*}
% \begin{bmatrix}
% r_{1i} &  r_{1n} \\ \vdots \\ r_{Ti} & r_{Tn}
% \end{bmatrix}
% \end{align*}

Bivariate return distribution:
\begin{align*}
\mu=
\begin{bmatrix}
\mu_{i} \\ \mu_{n}
\end{bmatrix}
\quad
\Sigma=
\begin{bmatrix}
\sigma_{i}^2 & \sigma_{ni}
\\
\sigma_{in} & \sigma_{n}^2
\end{bmatrix}.
\end{align*}

% sample counterparts
Any quantity $x$ when denoted as $\hat{x}$ simply means its sample counterpart, e.g. $\hat{\mu}$ denotes the sample counterpart of $\mu$.

% Differences
Define the difference of returns as $r_{td} = r_{ti} - r_{tn}$ with moments:
\begin{align*}
E(r_{td}) &= \mu_{d} = \mu_{i} - \mu_{n} 
	\\
V(r_{dt})&= E[(r_{ti} - r_{tn})^2] - [E(r_{ti} - r_{tn})]^2
\\
V(r_{dt})&= E(r^2_{ti}) + E(r^2_{tn}) - 2E(r_{ti}r_{tn}) - \mu_{i}^2 + \mu_{n}^2 - 2\mu_{i}\mu_{n}
\\
V(r_{dt}) &= V(r_{ti}) + V(r_{tn}) - 2Cov(r_{ti},r_{tn}) 
\\
\sigma^2_{d} &= \sigma^2_{i} + \sigma^2_{n} - 2\sigma_{in}
\end{align*}

% Classical Test
What we want to test is if the difference in means is different from zero statistically significant.
\begin{align*}
	H_{0}: \, \hat{\mu}_{d} = 0
	\text{ vs. }
	H_{1}: \, \hat{\mu}_{d} \neq 0.
\end{align*}

\textbf{The classical $T$-test:}

Define the test statistic as:
\begin{align}
\Theta &= T^{1/2} \frac{\hat{\mu}_{d}}{\hat{\sigma}_{d}} = 
 T^{1/2} \frac{\hat{\mu}_{i} - \hat{\mu}_{n}}{\hat{\sigma}^2_{i} + \hat{\sigma}^2_{n} - 2\hat{\sigma}_{in}}.
\end{align}
Further, let's define $t_{\lambda}(k)$ as the $\lambda$-quantile of $t(k)$, or the $t$ distribution with $k$ degrees of freedom.
The test rejects $H_{0}$ at a significance level $\alpha$ iff:
\begin{align*}
	|\Theta| > t_{1-\alpha/2}(T-1).
\end{align*}


%%%%%%%%%%%%%%%%%%%%%%%%%%%%%%%%%%%%%%%%%%%%%%%%%%%%%%
% REFORMULATION
\clearpage
\section{Reformulation}

What we are going to test is whether the difference in mean, $\Delta$, is zero or not.
In other terms:
\begin{align*}
	H_{0}: \Delta=0 \text{ vs. } H_{1}: \Delta \neq 0.
\end{align*}

Define $\Delta$ as a function of $v = (\mu_{i}, \mu_{n})'$:
\begin{align*}
\Delta = f(v) = \mu_{i} - \mu_{n}.
\end{align*}

The gradient of the function $f(v)$ is:
\begin{align}
\nabla' f(v) =
\left( 
\pdf{f(v)}{\mu_{i}},
\pdf{f(v)}{\mu_{n}}
\right) = \left( 1, -1 \right).
\end{align}

Now, assume that:
\begin{align}
	T^{1/2}(v - \hat{v}) \to^d N(0, \Psi),
\end{align}
where $\Psi$ is an unknown symmetric PSD matrix.

If we apply a function on the vector $v$ of parameters, the Taylor expansion (Delta method) implies:
\begin{align*}
	T^{1/2}[f(v) - f(\hat{v})] \to^d N \left( 0; \nabla'f(v) \Psi \nabla f(v)  \right).
\end{align*}

Well, we will use $f(\cdot)$ as defined earlier, and we denote $f(v) = \Delta$, so we have:
\begin{align*}
T^{1/2}(\Delta - \hat{\Delta}) 	\to^d N \left( 0; \nabla'f(v) \Psi \nabla f(v)  \right).
\end{align*}

Now, if a consistent estimator $\hat{\Psi}$ of $\Psi$ is available, then $se(\hat{\Delta})$ is given by:
\begin{align}
	se(\hat{\Delta}) = \sqrt{T^{-1} \nabla'f(\hat{v}) \hat{\Psi} \nabla f(\hat{v})}.
\end{align}

% 
\begin{align*}
\nabla'f(\hat{v}) \hat{\Psi} \nabla f(\hat{v})&=
\begin{bmatrix} 1 & -1 \end{bmatrix}
\begin{bmatrix}
\hat{\psi}_{11} &  \hat{\psi}_{12}
\\            
\hat{\psi}_{21} &  \hat{\psi}_{22}
\end{bmatrix}
\begin{bmatrix} 1 \\  -1 \end{bmatrix}
\\
&= \hat{\psi}_{11} + \hat{\psi}_{22} -2\hat{\psi}_{12}
\end{align*}

If we use $\hat{\Psi}$ as the sample covariance matrix:
\begin{align*}
\nabla'f(\hat{v}) \hat{\Psi} \nabla f(\hat{v})= \hat{\sigma}_{1}^2 + \hat{\sigma}_{22}^{2} -2\hat{\sigma}_{12}.
\end{align*}
And we have the same case as the classical $t$-test.
So we use HAC consistend methods to estimate $\hat{\Psi}$.

%%%%%%%%%%%%%%%%%%%%%%%%%%%%%%%%%%%%%%%%%%%%%%%%%%%%%%
% LOG REFORMULATION
\clearpage
\section{Log Reformulation}

Reformulating the problem with a log transformation: 
\begin{align}
\Delta = \log(\Theta) =
\log(1+\mu_{i}) - \log(1+\mu_{n})  =
\log[(1+\mu_{i})/(1+\mu_{n})].
\end{align}

\noindent
{\footnotesize
\textbf{OBS:} Log transformation, see Efron and Tibshirani 1993, sec 12.6.}
\vspace{1 em}

What we are going to test is whether the difference in log variances, $\Delta$, is zero or not.
In other terms:
\begin{align*}
	H_{0}: \Delta=0 \text{ vs. } H_{1}: \Delta \neq 0.
\end{align*}

Define $\Delta$ as a function of $v = (\mu_{i}, \mu_{n})'$:
\begin{align*}
\Delta &= f(v) =
\log{(1+\mu_{i})} - \log{(1+\mu_{n})} =
\log\left( {\frac{1+\mu_{i}}{1+\mu_{n}}} \right).
\end{align*}

The gradient of the function $f(v)$ is:
% gradient
\begin{align}
\nabla' f(v) &=
\left( 
\pdf{f(v)}{\mu_{i}},
\pdf{f(v)}{\mu_{n}}
\right) =
\left( 
\frac{1}{1+ \mu_{i}},
\frac{-1}{1+ \mu_{n}}
\right).
\end{align}

Now, assume that:
\begin{align}
	T^{1/2}(v - \hat{v}) \to^d N(0, \Psi),
\end{align}
where $\Psi$ is an unknown symmetric PSD matrix.

If we apply a function on the vector $v$ of parameters, the Taylor expansion (Delta method) implies:
\begin{align*}
	T^{1/2}[f(v) - f(\hat{v})] \to^d N \left( 0; \nabla'f(v) \Psi \nabla f(v)  \right).
\end{align*}

Well, we will use $f(\cdot)$ as defined earlier, and we denote $f(v) = \Delta$, so we have:
\begin{align*}
T^{1/2}(\Delta - \hat{\Delta}) 	\to^d N \left( 0; \nabla'f(v) \Psi \nabla f(v)  \right).
\end{align*}

Now, if a consistent estimator $\hat{\Psi}$ of $\Psi$ is available, then $se(\hat{\Delta})$ is given by:
\begin{align}
	se(\hat{\Delta}) = \sqrt{T^{-1} \nabla'f(\hat{v}) \hat{\Psi} \nabla f(\hat{v})}.
\end{align}

% 
\begin{align*}
\nabla'f(v) \Psi \nabla f(v)=
\begin{bmatrix} \frac{1}{1+ \mu_{i}} & \frac{-1}{1+ \mu_{n}} \end{bmatrix}
\begin{bmatrix}
\psi_{11} &  \psi_{12}
\\
\psi_{21} &  \psi_{22}
\end{bmatrix}
\begin{bmatrix} \frac{1}{1+ \mu_{i}} \\ \frac{-1}{1+ \mu_{n}} \end{bmatrix}
\end{align*}

\begin{align*}
\nabla'f(v) \Psi \nabla f(v)=
	\frac{\psi_{11}}{(1+\mu_{i})^2} +
	\frac{\psi_{22}}{(1+\mu_{n})^2} -
	\frac{2\psi_{21}}{(1+\mu_{i})(1+\mu_{n})} 
\end{align*}

%%%%%%%%%%%%%%%%%%%%%%%%%%%%%%%%%%%%%%%%%%%%%%%%%%%%%%
%   Bibliografia
\clearpage
\bibliographystyle{authordate3} % bibliography style
\renewcommand\bibname{REFERENCES} 
\bibliography{../refs.bib}

%%%%%%%%%%%%%%%%%%%%%%%%%%%%%%%%%%%%%%%%%%%%%%%%%%%%%%
\end{document}
